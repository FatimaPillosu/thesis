%%%%%%%%%%%%%%%%%%%%%%%%%%%%%%%%%%%%%%%%%%%%%%%%%%%%%%
\chapter{General discussions}
\label{general_discussions}
\graphicspath{{chapter_08/figures}{chapter_08/tables}}
%%%%%%%%%%%%%%%%%%%%%%%%%%%%%%%%%%%%%%%%%%%%%%%%%%%%%%


Flash \marginpara{Escalating flash flood impacts and climate-change-induced intensification of flash flood risk demand urgent advances in forecasting capabilities} floods represent the deadliest and most devastating form of natural hazard worldwide, causing over 5,000 fatalities annually and accounting for approximately 85\% of global flood incidents. Recent catastrophic events, such as those including (but not limited to) the flash floods in Spain, Libya, Germany, Central Asia, and Brazil, have claimed hundreds of lives, thousands of injuries, and billions in economic losses. As climate change continues to intensify the frequency and severity of extreme rainfall - including in historically low-risk regions - the development of accurate, timely, and globally accessible flash flood predictions has become a critical priority for disaster risk reduction. WMO has identified flash floods as one of its top priority natural hazards. Meanwhile, the UN's 'Early Warnings for All' initiative, launched in 2022 with the ambitious goal of protecting every person on Earth with early warning systems by 2027, places flash floods at the forefront of its agenda.

Significant \marginpara{Thesis aim: addressing technical barriers to medium-range predictions of areas at risk of flash floods over a continuous global domain to provide a viable pathway for protecting vulnerable communities worldwide} technical and methodological obstacles persist in developing medium-range flash flood forecasts over continuous global domains. The overall aim of this thesis was to address the critical gap in global flash flood early warning capabilities and to demonstrate how data-driven approaches, combined with global numerical weather prediction models, can provide a viable pathway for protecting vulnerable communities worldwide, particularly those in data-scarce regions where traditional forecasting approaches face significant limitations. These three interconnected research objectives have been addressed in the main analysis chapters of this thesis:

\begin{tcolorbox}[
  colframe=colour_chapter5,  
  colback=white,           
  sharp corners,        
  boxrule=2mm,          
  left=0mm,             
  right=0mm,            
  toprule=0mm,          
  bottomrule=0mm,       
  rightrule=2mm        
]
{\color{colour_chapter5} {\setlength{\parindent}{1.0em} Chapter 5: depart from the traditional rainfall-to-rainfall verification approach and adopt, instead, a \textit{flash-flood-focused verification framework} that directly compares rainfall forecasts with flash flood impact reports to answer research question n.1 (RQ1) "Can post-processed global NWP rainfall forecasts successfully identify areas at risk of flash floods up to medium-range lead times?"}}
\end{tcolorbox}

\begin{tcolorbox}[
  colframe=colour_chapter6,  
  colback=white,           
  sharp corners,        
  boxrule=2mm,          
  left=0mm,             
  right=0mm,            
  toprule=0mm,          
  bottomrule=0mm,       
  rightrule=2mm        
]
{\color{colour_chapter6} {\setlength{\parindent}{1.0em} Chapter 6: develop data-driven models that integrate hydro-meteorological variables from global reanalysis and global medium-range NWP forecasts, and flash flood impact reports to predict areas at risk of flash floods, from short (i.e., day 1) to medium-range lead times (i.e., day 5), to answer research question n.2 (RQ2) "Are medium-range data-driven hydro-meteorological predictions of areas at risk of flash floods feasible with global reanalysis, forecasts, and impact flash flood reports?"}}
\end{tcolorbox}

\begin{tcolorbox}[
  colframe=colour_chapter7,  
  colback=white,           
  sharp corners,        
  boxrule=2mm,          
  left=0mm,             
  right=0mm,            
  toprule=0mm,          
  bottomrule=0mm,       
  rightrule=2mm        
]
{\color{colour_chapter7} {\setlength{\parindent}{1.0em} Chapter 7: assess how varying spatial coverage and data density scenarios may influence training strategies when creating global predictions with regionally-trained data-driven models, to answer research question n.3 (RQ3) "How does coverage-density trade-off influence training data strategies to develop predictions of areas at risk of flash floods over a continuous global domain?"}}
\end{tcolorbox}

This chapter discusses the outcomes, implications, and limitations of the research from each of these three main analysis chapters, synthesises the broader contributions to flash flood prediction science, and presents recommendations for advancing global early warning capabilities against flash floods that could ultimately save thousands of lives every year and support UN's vision of global early warning coverage - for flash floods.


%%%%%%%%%%%%%%%%%%%%%%%%%%%%%%%%%%%%%%%%%%%%%%%%%%%%%%%%%%
\section{Development of a flash-flood-focused verification framework for predictions of areas at risk of flash flood against flash flood impact reports}

In Chapter 5, we departed from the traditional rainfall-to-rainfall verification paradigm and developed a flash-flood-focused verification framework that directly compares rainfall forecasts with flash flood impact reports. This approach recognised the fundamental non-linearity between precipitation and flash flood occurrence, establishing a more direct pathway for assessing the operational utility of global NWP rainfall forecasts. We systematically evaluated ERA5-ecPoint post-processed rainfall forecasts against flash flood reports from NOAA's Storm Event Database over the CONUS for the period 2021-2024, examining forecast performance across multiple return period thresholds (1- to 100-year) and lead times (day 1 to day 5). This methodology enabled us to establish whether global NWP models could successfully identify areas at risk of flash floods at operationally relevant medium-range timescales.

\subsection{Key insights and contributions}

This research represents a conceptual shift from verifying global NWP rainfall forecasts for only \textit{rainfall accuracy} to verifying \textit{flash flood risk identification capability}. Our results demonstrate that global NWP models, specifically ECMWF's ERA5 when post-processed through the ecPoint technique, can successfully identify areas at risk of flash floods up to five-day lead times. % Discuss the results here.

The framework's most significant contribution is establishing that high return periods, e.g., > 20 years, can serve as an effective proxy for flash flood risk, achieving frequency bias values closest to unity when compared against observed events. This finding provides operational forecasters with a clear, implementable threshold for issuing warnings based on globally available data.

% add that this framework, even though has been built for flash floods, it can be used for any other hazard where IMPACT REPORTS may be available. So the framework is fully transferable to other hazards.

\subsection{Limitations}

Several critical limitations constrain the generalisability and operational implementation of our findings. Foremost, the verification was conducted exclusively over the CONUS, leveraging the exceptional quality and spatial coverage of NOAA's Storm Event Database. The transferability of both the verification framework and the established performance metrics to data-sparse regions remains untested, particularly where impact reporting systems lack the standardisation and comprehensiveness of the US system. The inherent biases in impact reporting—including population density effects, diurnal reporting variations, and socio-economic disparities in hazard documentation—introduce systematic uncertainties that our framework cannot fully quantify.

The rainfall-only approach, whilst demonstrating unexpected predictive skill, fundamentally neglects the hydrological processes that modulate rainfall-runoff transformation. Antecedent soil moisture conditions, land surface characteristics, urbanisation extent, and drainage network properties all influence whether a given rainfall event triggers flash flooding, yet these factors remain unaccounted for in our methodology. The fixed percentile threshold approach (99th percentile) represents a further simplification that may not optimally capture flash flood risk across diverse hydroclimatic regimes—what constitutes extreme rainfall in arid regions differs markedly from humid environments.

Computationally, the ecPoint post-processing technique demands significant resources through its weather regime analogue matching process, potentially limiting operational implementation in resource-constrained environments. Finally, our fundamental assumption that absence of reports equates to absence of events inevitably inflates false alarm rates, presenting a systematically pessimistic view of forecast performance that may not reflect true predictive capability.

\subsection{Future research directions}

Advancing this verification framework requires addressing both methodological refinements and geographical expansion. Immediate priorities should include extending the verification to other data-rich regions (Europe using ESWD, Japan, Australia) to establish the global applicability of both the framework and the performance benchmarks. This geographical expansion would enable the identification of region-specific optimal percentile thresholds and the quantification of how verification metrics vary across different impact reporting systems.

Methodologically, developing adaptive threshold selection algorithms that account for seasonal and regional hydroclimatic variability could substantially improve forecast reliability. Machine learning approaches could identify optimal percentile thresholds as functions of season, antecedent conditions, and geographical location, moving beyond the fixed 99th percentile approach. Integration of the verification framework with lightweight hydrological proxies—such as antecedent precipitation indices, soil moisture percentiles from global products, or topographic wetness indices—could enhance predictive skill whilst maintaining computational efficiency suitable for global application.

Alternative post-processing techniques warrant investigation, particularly machine learning-based downscaling methods that could reduce computational demands whilst preserving or enhancing forecast quality. Deep learning architectures trained on the relationship between coarse-resolution NWP outputs and high-resolution precipitation observations could potentially replace the computationally intensive weather regime matching of ecPoint. Finally, developing bias correction methodologies for impact databases—potentially through integration of satellite-based flood detection, social media mining, and population density weighting—could address the systematic underreporting that affects verification metrics, providing more accurate assessments of true forecast performance.