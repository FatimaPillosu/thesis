\graphicspath{{Figures/Chapter_04}{Tables/Chapter_04}}

\chapter{Experimental design}


%%%%%%%%%%%%%%%%%%%%%%
\section{Introduction}
%%%%%%%%%%%%%%%%%%%%%%

This chapter details the proposed methodological framework to produce medium-range predictions of areas at risk of flash floods over a continuous global domain. The framework follows a three-component approach, where each component builds upon the findings of the previous one, ensuring methodological coherence. This methodology also enables quantification of uncertainty at each stage of the methodological framework.

The first methodological component addresses the fundamental challenge that high-quality rainfall forecasts do not necessarily translate directly to accurate flash flood predictions. It therefore assesses the capability of medium-range global rainfall forecasts to identify areas at risk of flash floods up to medium-range lead times by proposing a flash-flood-focused rainfall verification framework. This initial verification is essential as it establishes the baseline predictive capacity of rainfall predictions from state-of-the-art global NWP models to identify areas at risk of flash flood risk.The second methodological component develops a machine learning model to identify areas at risk of flash floods over a continuous global domain. This data-driven approach represents a departure from traditional physically-based hydrological modelling, which often struggles with the computational demands and parameter uncertainty inherent in global-scale flash flood prediction. By utilising the Storm Event Database observations from the United States as training data, coupled with hydrological parameters from ERA5 reanalysis and precipitation characteristics from ERA5-ecPoint, the model establishes relationships between meteorological-hydrological conditions and observed flash flood occurrences. This approach capitalises on the relative abundance of flash flood observations in the United States whilst developing a framework that can be applied globally.
The third methodological component extends the application of the data-driven model to medium-range forecasts, enabling an assessment of flash flood predictability across extended forecast horizons. This component synthesises the findings from the previous two methods, applying the machine learning model framework to longer-range forecast data to evaluate how predictive skill for flash flooding decays with increasing lead time. This evaluation is crucial for understanding the temporal limitations of actionable flash flood predictions.

The following sections will detail each methodological component, including data sources, processing techniques, model development procedures, and verification frameworks. Particular attention is given to the challenges of developing a globally applicable model from geographically limited training data, and to the methodological considerations required for meaningful extension to medium-range forecasts and spatial coverage over a continuous global domain.


