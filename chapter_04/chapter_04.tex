%%%%%%%%%%%%%%%%%%%%%%%%%%%%%%%%%%%%%%%%%%%%%%%%%%%%%%
\chapter{Experimental design}
\label{experimental_design}
\graphicspath{{chapter_04/figures}{chapter_04/tables}}
%%%%%%%%%%%%%%%%%%%%%%%%%%%%%%%%%%%%%%%%%%%%%%%%%%%%%%



\section{Development of a flash-flood-focused verification framework}

In this section, we define a \textit{flash-flood-focused verification framework to assess (short- and medium-range) predictions of areas at risk of flash floods}. This framework will be used throughout the thesis to assess the performance of the rainfall-based flash flood forecasting system and its multi-parameter counterpart provided here by both data-driven models, the short-range (computed with the short-range ERA5 and ERA5-ecPoint forecasts) and the medium-range (computed with the equivalent long-range forecasts up to day 5). The application of the flash-flood-focused verification framework to the identification of areas at risk of flash floods involving solely rainfall predictions is done in the recognition that a robust understanding of rainfall forecast performance in identifying areas at risk of flash floods is essential to benchmark the performance of more advanced prediction methods, e.g. data-driven approaches involving more parameters.

The objective verification framework uses flash flood impact observations as ground truth. As indicated in chapter \ref{datasets} in section \ref{storm_event_database}, this thesis will consider only the Storm Event Database over the CONUS. However, this method can be applied in other geographical regions where regional databases are available, such as over Europe (using the ESWD dataset) or a global domain (considering global datasets such as EM-DAT), while always acknowledging the biases that such impact databases might possess. 

Each flash flood report must first be grouped with all the reports belonging to the accumulation period for which the flash flood outlooks are provided. This thesis provides predictions of areas at risk of flash floods over 24-hourly accumulation periods starting at 00 UTC. Hence, this step's outcome consists of the list of \textit{point} flash flood reports happening between the 1st of January at 00 UTC and the 2nd of January at 00 UTC, between the 2nd of January at 00 UTC and the 3rd of January at 00 UTC, etc. 

Each point report in a specific accumulation period must then be located on the ERA5 grid-box where the event happened. In the Storm Event Database, the location of each flash flood event is provided by two latitude coordinates ("BEGIN_LAT" and "END_LAT") and two longitude coordinates ("BEGIN_LON" and "END_LON"). If BEGIN_LAT = END_LAT and BEGIN_LON = END_LON, the flash flood event is identified by a single point, and only the closest grid-box to that point is assigned the flash flood event. If BEGIN_LAT \neq END_LAT and BEGIN_LON \neq END_LON, a polygon is provided an all grid-boxes within that polygon get assigned the flash flood event. It is worth noting, that even when a polygon is provided, only one grid-box might contain the flash flood event is the radius of the event was not bigger than the ERA5's grid-boxes resolution, i.e., 31 km. This procedure creates \textit{gridded} fields of flash flood reports, where the values assigned to the grid-boxes indicate the number of flash flood reports occurred within the grid-boxes, e.g. a grid-box with 0 does not contain any flash flood reports for the considered accumulation period, while a grid-box with a value of 5 indicates that there are 5 flash flood reports within that grid-box, for the considered accumulation period.  





so novel metrics are developed to handle the differences between continuous forecasts and binary observations, while accounting for the unique challenges posed by the rare and localised nature of flash floods. This analysis quantifies explicitly the uncertainties associated with existing prediction systems, providing a transparent assessment of their reliability and skill, informing the subsequent development of data-driven methods, in particular highlighting areas where data-driven techniques can offer significant improvements. A case-study-based approach is also used to provide in-depth insights into forecast performance differences between large-scale and small-scale (localised) flash flood events. The study was conducted at a regional level over continental Ecuador. To ensure the transferability of the findings from the Ecuador's example, the verification framework is designed to be modular and adaptable. The assessment metrics and criteria are selected based on their applicability to different geographical contexts, and the verification architecture is built to accommodate a wide range of data sources and formats. 

