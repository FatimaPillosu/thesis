%%%%%%%%%%%%%%%%%%%%%%%%%%%%%%%%%%%%%%%%%%%%%%%%%%%%%%
\chapter{Experimental design}
\label{experimental_design}
\graphicspath{{chapter_04/figures}{chapter_04/tables}}
%%%%%%%%%%%%%%%%%%%%%%%%%%%%%%%%%%%%%%%%%%%%%%%%%%%%%%


\section{Development of a flash-flood-focused verification framework}

This study faces two main challenges. The first challenge consists in the definition of the verifying thresholds that will be used to assess the performance of the rainfall-based and multi-parameter-based, data-driven predictions of areas at risk of flash floods. The second challenge consists of the use of appropriate verification scores due to the intrinsic characteristic of the observational datasets as it is not possible to define univocally observational yes- and non-events. 

In this section, we define a \textit{flash-flood-focused verification framework to assess (short- and medium-range) predictions of areas at risk of flash floods}. This framework will be used throughout the thesis to assess the performance of the rainfall-based flash flood forecasting system and its multi-parameter counterpart provided here by both data-driven models, the short-range (computed with the short-range ERA5 and ERA5-ecPoint forecasts) and the medium-range (computed with the equivalent long-range forecasts up to day 5). The application of the flash-flood-focused verification framework to the identification of areas at risk of flash floods involving solely rainfall predictions is done in the recognition that a robust understanding of rainfall forecast performance in identifying areas at risk of flash floods is essential to benchmark the performance of more advanced prediction methods, e.g. data-driven approaches involving more parameters.

The objective verification framework uses flash flood impact observations as ground truth. As indicated in section \ref{storm_event_database}, this thesis will consider only the Storm Event Database over the CONUS. However, this method can be applied in other geographical regions where regional databases are available, such as over Europe (using the ESWD dataset) or a global domain (considering global datasets such as EM-DAT), while always acknowledging the biases that such impact databases might possess. 


\subsection{Observational fields}
The definition of the observational dataset follows the methodology proposed by \citep{Tsonevsky_2018} for severe weather, and \citep{Pillosu_2024} for flash floods in Ecuador. Each flash flood report must first be grouped with all the reports belonging to the accumulation period for which the flash flood outlooks are provided. This thesis provides predictions of areas at risk of flash floods over 24-hourly accumulation periods starting at 00 UTC. Hence, this step's outcome consists of the list of \textit{point} flash flood reports happening between the 1st of January at 00 UTC and the 2nd of January at 00 UTC, between the 2nd of January at 00 UTC and the 3rd of January at 00 UTC, etc. 

Each point report in a specific accumulation period must then be located on the ERA5 grid-box where the event happened. In the Storm Event Database, the location of each flash flood event is provided by two latitude coordinates ("BEGIN\_LAT" and "END\_LAT") and two longitude coordinates ("BEGIN\_LON" and "END\_LON"). If BEGIN\_LAT = END\_LAT and BEGIN\_LON = END\_LON, the flash flood event is identified by a single point, and only the closest grid-box to that point is assigned the flash flood event using the "nearest grid-box" method. If BEGIN\_LAT \ne END\_LAT and BEGIN\_LON \ne END\_LON, a polygon is provided in all grid-boxes within that polygon get assigned the flash flood event. It is worth noting, that even when a polygon is provided, only one grid-box might contain the flash flood event is the radius of the event was not bigger than the ERA5's grid-boxes resolution, i.e., 31 km. This procedure creates \textit{gridded} fields of flash flood reports, where the values assigned to the grid-boxes indicate the number of flash flood reports occurred within the grid-boxes, e.g. a grid-box with 0 does not contain any flash flood reports for the considered accumulation period, while a grid-box with a value of 5 indicates that there are 5 flash flood reports within that grid-box, for the considered accumulation period.

Because of the large size of the ERA5 grid-boxes (i.e., 31 km) and the large accumulation periods for the flash flood outlooks (i.e. 24 hours), any additional expansion in the area or reporting times of individual reports is probably unnecessary to account for uncertainties in the location and reporting times. It is possible to envision remote scenarios in which a report is close enough to the edge of the ERA5 grid-box or the outlook accumulation period that a location or reporting time error in the database could cause the mislocation of such flash flood event, but this scenario has been established to be for ~0.1\% of the total number of the reports considered in this thesis, so no further methodologies will be considered here to account for uncertainties in the location or reporting time in the Storm Event Database. 

Figure \ref{fig:sed_reports} shows an example of the post-processed flash flood reports from the Storm Event Database.

\begin{figure}[htbp]
\centering
\includegraphics[scale=0.65]{sed_reports.png}
\caption{\textbf{Example of post-processed reports from the Storm Event Database.} Panel (a) shows the timeseries of the counts of all flood types recorded in the database from January 1950 to December 2023 (in grey), the counts of all flash floods recorded (in dark red), and the counts of all flash floods records including lat/lon coordinates (in red). Panel (b) shows the timeseries of point (in red) and gridded (in black) flash flood reports in 2021, accumulated over 12-hourly accumulation periods ending at 00 and 12 UTC. Panel (c) shows the spatial distribution of point flash flood reports (in red) during the 12-hourly accumulation period ending on 2021-09-02 at 00 UTC (i.e. for storm Ida). The zoomed in area shows an example of the number of point flash flood reports (in red) assigned to the nearest grid-box. Those grid-boxes containing at least one point flash flood reports is coloured in black. Panel (d) shows the spatial distribution of the observed frequency of flash flood events in each grid-box between 2005 and 2023. A distinction between four main areas is applied: the probabilities in the north-west, north-east, south-east and south-west are indicated, respectively, in orange, turquoise, cyan, and green. The probabilities are discretised in equal to 0\% (in white for all the regions), between 0-0.1\%, 0.1-1\%, and 1-10\% (in different tones of the colour for the corresponding region). Panel’s (d) insert shows a piechart representing the proportion of gridded flash flood reports in each considered region.}
\label{fig:sed_reports}
\end{figure}


\subsection{Forecast fields}

Given a forecast value at a grid-box, a flash flood event is considered as a yes-event if the forecast exceeds a specific event threshold (whose definition is explained in section \ref{verifying_rainfall_threshold}). The forecast fields are then created by assigning the value 1 to all those grid-boxes where the forecasts exceed the considered threshold; otherwise, the value 0 is assigned. 

\subsection{Definition of the verifying thresholds}
\label{verifying_rainfall_threshold}
For the rainfall-based predictions of areas at risk of flash flood, a rainfall-based threshold must be used. For the verification of rainfall-based predictions of areas at risk of flash floods, flash-flood-triggering rainfall totals might be known for the region of interest, so that they can be used as thresholds to define the flash flood events to verify. In this thesis, such rainfall thresholds are not know for the CONUS, and therefore, must be computed from data. If point rainfall observations are available (e.g., rain gauges or radars), one can create the distribution of the observed flash-flood-triggering rainfall totals, and the verifying rainfall thresholds (VRT) values would then correspond to the specific percentiles of the distribution. The higher the percentile, the higher the magnitude of the VRT, and the higher the severity level of flash flood events considered in the objective verification analysis. This approach requires high-density rainfall observations, in both space and time, to capture the localised extreme rainfall totals that trigger flash floods \citep{Haiden_2016, RamosFilho_2021}. In the absence of a suitable observational network, the VRT values can be defined only from gridded rainfall products such as reanalysis e.g., ERA5 \citep{Hersbach_2020}, reforecasts \citep{Hamill_2006b}, or blended rainfall observations provided on a grid such as MSWEP \citep{Beck_2019} or GPCP \citep{Adler_2018}. These datasets, however, tend to underestimate rainfall extremes because of their coarse spatial resolution \citep{Tapiador_2019}. In the absence of more suitable gridded datasets, \citet{Pillosu_2024} proposed a methodology to compute regional point rainfall climatologies using 1 year of short-range (day 1) ecPoint rainfall forecasts. This approach, however, has the disadvantage to use only 1 year of forecasts, which means that the resulting climatologies will inevitably be affected by the climatology of that specific year instead of the general climatology of the region that they are supposed to represent. Moreover, as this climatology requires flash flood observations to be defined, and such observations tend to have an even lower resolution than rainfall forecasts, there is the need to create climatologies for large domains, rather than on a grid-box scale. For example, \citet{Pillosu_2024} created only two verifying rainfall thresholds for Ecuador, one for the coastal and one for the Andean region. While this disaggregation is better than having one single value for an entire country, regional thresholds might not be able to capture flash-flood-triggering rainfall thresholds over different micro-climates. This would not allow an appropriate disaggregation of the performance of the flash flood forecasts in different regions, under different (hydro-meteorological) conditions. Hence, in this thesis, we propose the definition of the rainfall VRTS from a climatology built with ERA5-ecPoint rainfall estimates, as they have been shown to represent point-rainfall climatologies around the world reliably \citep{Pillosu_2025a} and they would allow us to develop a grid-box scale rainfall climatology to be used as thresholds for the rainfall-based predictions of areas at risk of flash floods. Hence, an ERA5-ecPoint rainfall climatology has been computed over the WMO recommended 30-year period 1991-2020 \citep{WMO_2017}. Table \ref{} shows the percentiles and corresponding values as x-year return periods that have been considered as flash-flood-triggering rainfall events during the verification.


\subsection{Objective verification}

The objective verification carried out in this thesis is based of verification scores computed through a probabilistic contingency table. We follow \citet{Pillosu_2024} methodology for the population of the contingency tables. Stationary observations (i.e. provided by instruments installed at a specific location, such as rain gauges, provide timeseries of yes- and non-events recorded at the location where the instrument was installed. THus, all four elements in the contingency table can be quantified. Non-stationary observations record only yes-events at the location where the event occurred. As a result, it is impossible to answer the question "if there are no reports in an area, is it because an event happened but nobody reported it, or because there was no event to report?". Some studies facing the same problem because they use non-stationary observations such as impact reports, verify only yes-events with the caveat that only quadrant I (i.e. hits) and III (i.e. misses) of the contingency table can be populated \citep{Robbins_2018}. Instead, this study followed the method proposed by \citet{Tsonevsky_2018}, which allows to populate all quadrants of the contingency table. This method assumes that a non-reports corresponds to a non-event. 

Hence, the contingency tables are built by examining overlapping grid boxes in correspondent observational and forecast fields. When both grid-boxes are assigned a value of 1 or 0, they count as a hit or a correct negative, respectively. When a grid-box in the observational field is assigned a value of 1, and the corresponding grid-box in the forecast field is assigned a value of 0, it counts as a miss. It counts as a false alarm if the opposite occurs. 


\subsection{Properties of probabilistic forecasts}

Reliability and discrimination ability are desirable properties of ensemble forecasts, and both are defined against a verifying threshold \citep{Jolliffe_2012, Wilks_2020}. Reliability measures whether the chosen verifying threshold is predicted with probabilities that mirror the frequency with which the considered event is observed. Discrimination measures the forecasts' ability to distinguish situations that lead to events exceeding the verifying threshold from those that do not, appraising the existence of a signal in forecasts when an event materialises. In this thesis, we consider two types of scores to analyse reliability and discrimination ability: summary and breakdown scores. Summary scores show the overall reliability and discrimination ability of the forecasts, while the breakdown scores provide detailed insights into how reliability and discrimination ability relate to the full distribution of probabilities. 


\subsubsection{Reliability}

To assess the overall reliability of the forecasts, we will consider the frequency bias (FB) to evaluate the overall reliability of the predictions of areas at risk of flash floods. The frequency bias was determined by dividing the total number of yes-events in the forecasts by the total number of yes-events in the observations. FB values range from 0 to $+\infty$. FB = 1 indicates perfect calibration, while scores greater or smaller than 1 indicate, respectively, over- and under-prediction of the observed yes-events. It is worth noting that FB measure the overall ratio of forecasts events to observed events and is not a measure of forecast sill. As such, it can provide a score of 1 when there are compensating error. Moreover, FB might show large overestimations if the observed event is heavily underreported. This is our case as explained in section \ref{fig:sed_reports}.

To break down the reliability of the forecasts over the full distribution of probabilities, reliability diagrams are used. They plot the relative forecasts probability of an event against its corresponding relative observational frequency, indicating how reliable the forecast probabilities are at different classes. For perfect forecasts, when the forecasts show x\% probability of occurrence, observations should meet the criteria x\% of the time, so that the reliability curve lies on the diagonal. If the reliability diagram is above the diagonal for a specific forecast probability, those forecasts are under-predicting the likelihood of observing a yes-event. If it lies below the diagonal, the is over-prediction. When analysing reliability diagrams, it is also important to know the frequency distribution of forecasts issued with specific probabilities. For example, the small probability thresholds (within the red box in the figure example) are the most important when considering high verifying thresholds because the sample of forecasts exceeding the verifying threshold with high probabilities is rather small. For this reason, reliability diagrams should be accompanied by sharpness diagrams, which plot the absolute frequency of forecasts of different probabilities. 


\subsubsection{Discrimination ability}

Relative Operating Characteristic (ROC) curves are built from 2x2 contingency tables (Table \ref{}), quantifying hits (H) misses (M) false alarms (FA), and correct negatives (CN). Hit rates (HR) and false alarm rates (FAR) are computed, respectively from equations:




HRs are mapped (Y-axis) against FARs (X-axis) in a unit square. The form of the ROC curve shows how HRs vary with FARs as one systematically lowers the thresholds probability at which it is assumed that an event has technically been forecast to happen (i.e. 100\% of probability the bottom left corner to 0\% probability at the top right corner). The values of the geometrical area under the ROC curve (AROC) provide a summary measure of the discrimination ability across all probability thresholds. The plot of AROC values for different lead times allows us to compare the discrimination ability between the rainfall-based predictions of areas at risk of flash floods, and the multi-parameter-based, data-driven predictions, (both at short- and medium-range lead times). Perfect discrimination is obtained when only HRs grow and FARs remain zero. It is represented by a ROC curve that rises along the Y-axis from the bottom left corner of the unit square to the top-left corner and moves straight to the top-right corner. In this case, the AROC is equal to 1. If HRs and FARs grow at the same rate, the forecasts have no discrimination ability (as a climatological forecast). In this case, the ROC curve lies along the diagonal and AROC equals 0.5. 

How ROC curves and AROCs are computed can impact the interpretation of forecasts discrimination ability. For rainfall-based predictions, the ROC curves will be built for incremental decision thresholds that are materially assessable from the real ensemble configuration. In this way, we can estimate the "real" forecast discrimination ability \citep{wilks_statistical_2020}. Probability thresholds are determined by considering the full discretisation ability in the ensemble (e.g., 99 members in the case of ERA5-ecPoint). This ensures that the ROC curves are as complete as possible \citep{Bouallegue_2022}. The number of thresholds corresponds, therefore, to the number of members exceeding the verifying rainfall threshold so that for an enmble of size M, maximum discretisation is achieved by M+1 probability thresholds (i.e., 0, 1/M, 2/M, ...., M/M=1). For the data-driven forecasts, where the forecast probability is provided by the model with continuous numbers between 0 and 1, the decision (probability) thresholds are defined by the user. In this thesis, a discretisation of 0.01 (equivalent to 1\%) and 0.001 (equivalent to 0.1\%) will be considered given the low frequency observed of flash flood events in the observational database. The ROC curve is then built by straight segments joining successive points. It is then completed by joining that last meaningful point with a straight line in the top-rigth corner of the unit square. FOr rare events, the points of a ROC curve cluster in the graph's bottom left corner and completing the ROC with a straight line might give the impression that part of the curve is missing \citep{Casati_2008}. How much the curve appears incomplete depends on the ensemble size and the base rate of the event. The area under the ROC curve (AROC) will be computed using a trapezoidal approximation by adding the areas of single trapeziums formed by the straight lines between consecutive points in the ROC curve \citep{Bouallegue_2022}. 

From what written before, the ROC curves represent the breakdown measure of discrimination ability as it will be possible to examine the values of HRs and FARs at different decision (probability) thresholds. AROC will represent instead the overall measure of discrimination ability.

