% Adding a coloured vertical edge to the pages in the chapter
\ClearShipoutPicture
\AddToShipoutPicture{%
  \AtPageLowerLeft{%
    \checkoddpage
    \ifoddpage
      \begin{tikzpicture}[remember picture,overlay] % Odd page → right edge
        \draw[line width=80pt, colour_chapter7] 
             (\paperwidth,0) -- (\paperwidth,\paperheight);
      \end{tikzpicture}%
    \else
      \begin{tikzpicture}[remember picture,overlay] % Even page → left edge
        \draw[line width=80pt, colour_chapter7] 
             (0,0) -- (0,\paperheight);
      \end{tikzpicture}%
    \fi
  }%
}

%%%%%%%%%%%%%%%%%%%%%%%%%%%%%%%%%%%%%%%%%%%%%%%%%%%%%%%%%%%%%
\chapter{Towards predictions over a continuous global domain: 
global implementation of regionally-trained models}
\label{regional_to_global_training}
\graphicspath{{chapter_07/figures}{chapter_07/tables}}
%%%%%%%%%%%%%%%%%%%%%%%%%%%%%%%%%%%%%%%%%%%%%%%%%%%%%%%%%%%%%

\underline{\textbf{Authors' contribution for this chapter:}} Fatima M. Pillosu designed the study, with advice from Hannah Cloke and Christel Prudhomme, obtained the datasets, carried out the analysis, and led the writing of the manuscript. All authors assisted with writing the manuscript. Overall, 90\% of the writing was undertaken by Fatima M. Pillosu.

\vspace{\baselineskip}

\section*{PREFACE}
\addcontentsline{toc}{section}{PREFACE}



\clearpage

\section*{ABSTRACT}
\addcontentsline{toc}{section}{ABSTRACT}

\clearpage



%%%%%%%%%%%%%%%%%%%%%%
\section{Introduction}

The aspiration to develop predictions of areas at risk of flash floods over a continuous global domain confronts a fundamental paradox that extends beyond technical challenges to encompass critical questions of equity in disaster risk reduction and preparedness. Flash floods represent one of the most devastating natural hazards globally, affecting populations in the Global North and Global South equally. However, the observational infrastructure necessary for developing predictive models that can help the population prepare and mitigate the risk against flash floods remains concentrated in a small subset of wealthy nations. This disparity violates the principle that all populations, regardless of their location or economic status, should have access to life-saving flood warnings — a goal central to the UN's 'Early Warnings for All' initiative. Moreover, as flash floods typically occur in poorly gauged catchments, this further reduces the number of observations available for model development and post-hoc event analysis even in data-rich regions in the Global North, thereby hindering our understanding of flash flood generation mechanisms. This inequitable distribution of observational capacity to assess flash flood occurrence creates an urgent need for innovative approaches that can transcend the limitations of traditional catchment-specific modelling paradigms to develop predictive models over a continuous domain able to truly cover all populations around the globe.

The emergence of data-driven approaches in large-sample hydrology has demonstrated remarkable success in learning complex relationships between hydro-meteorological variables and flood occurrence. Many data-driven applications are also now applied to predict areas at risk of flash floods or river discharge in flashy catchments. However, such applications remain primarily at the catchment level due to the aforementioned severe paucity of observations suitable for predicting flash flood events. There exist only a few examples of prediction systems at a larger scale (e.g., national).

Recent advances in transfer learning and domain adaptation offer a transformative approach to this challenge. Rather than requiring comprehensive local observations for model training, one could train a data-driven model to learn generalisable hydro-meteorological relationships from data-rich regions and subsequently deploy the model to create predictions over a continuous global domain, provided that hydro-meteorological variables from global NWP models are used. The key insight underlying this approach is that whilst specific catchment characteristics may vary globally, the fundamental physical processes governing flash flood generation — the interaction between intense precipitation, antecedent soil moisture, topography, and land surface characteristics — exhibit sufficient commonality to enable knowledge transfer across different regions.

The development of such transferable models faces a critical trade-off between spatial coverage and data density. Models trained on high-density observations from limited geographical regions may capture local flash flood dynamics with high fidelity, but may fail to generalise to regions with different climatic regimes, topographies, or land use patterns. Conversely, models trained on sparse global datasets may achieve broader applicability but at the cost of reduced accuracy in the overall identification of areas at risk of flash floods. 

This chapter addresses Research Question 3 of this thesis by systematically investigating how the coverage-density trade-off influences training strategies for developing predictions of areas at risk of flash floods over a continuous global domain. Building upon the data-driven models developed in Chapter \ref{data_driven_flash_floods_short_medium_range}, we explore three distinct approaches to training data selection through sensitivity analysis over the CONUS domain. Approach 1 employs random spatial sampling that maintains geographical coverage across the full CONUS domain whilst reducing overall observation density. This approach simulates conditions analogous to global model training using sparse impact reports from databases such as EM-DAT. The model is exposed to the entire global domain with impact reports distributed across the globe, but at an extremely low observational density—tens of times lower than the already sparse 0.27\% coverage of the Storm Event Database. Approach 2 utilises regionally-constrained training with global visibility. The model trains on flash flood observations from only a subset of the spatial domain whilst maintaining exposure to the full domain during training. This approach simulates conditions where the model learns from the full global domain but receives impact reports only from data-rich regions such as the USA and Europe. Consequently, the model learns that certain regions appear to have no flash flood occurrences. This approach investigates whether such training affects the model's ability to predict areas at risk of flash floods in regions where no events have been observed. Approach 3 implements domain-exclusion training, where the model trains exclusively on a geographically restricted subset without exposure to regions lacking observations. This approach simulates conditions where the model only encounters geographical domains with higher observational density, completely excluding areas with no observations or poor coverage. This method examines the implications for flash flood risk predictions in the excluded regions and tests the model's capacity for spatial extrapolation to entirely unseen geographical areas.

% Complete the ending for the introduction.


%%%%%%%%%%%%%%%%%%%%%%%%%%
\section{Methods and Data}

\subsection{Training approaches}
The methodological framework examines three distinct training strategies to determine the optimal approach for developing flash flood predictions across a continuous global domain, considering heterogeneous data availability. 

This \marginpara{Training approach 1 (TA1): randomly reduced density} approach tests whether training over the full CONUS domain remains effective when flash flood observations are systematically reduced. Flash flood reports from the Storm Event Database were randomly sampled at three levels: 90\%, 50\%, and 10\% of the original dataset. The sampling was applied uniformly across the entire domain, maintaining the spatial distribution whilst reducing observation density. This approach simulates the scenario of training a global model with sparse but spatially distributed observations.

The \marginpara{Training approach 2 (TA2): sparse regional coverage} second approach maintains training over the full CONUS domain but restricts flash flood observations to specific regions. Unlike Approach 2, the model receives hydro-meteorological data from the entire domain during training but only has access to flash flood labels in the selected region. The non-selected regions contribute only negative samples (non-flood events) to the training dataset. This configuration simulates the realistic scenario of developing a global model where flash flood reports are available only from certain countries or regions, whilst meteorological data has global coverage.

The \marginpara{Training approach 3 (TA3): domain-restricted training} third approach examines whether models trained on high-quality observations from limited geographical regions can effectively predict flash floods in areas where they have never observed any events. The CONUS domain was divided into four regions: East (east of 98°W), West (west of 98°W), North (north of 37°N), and South (south of 37°N). For each configuration, the model was trained using flash flood observations from only one region, with the remaining regions containing no observations during training. The trained model was then applied to predict flash floods across the entire CONUS domain. This approach directly addresses the scenario where certain parts of the world have excellent observational infrastructure, whilst others have none.



\subsection{Training data configuration}
For all approaches, the baseline configuration consists of the full Storm Event Database for the CONUS from 2016-2020, containing 247,953 flash flood reports. The hydro-meteorological features remain consistent with those developed in Chapter 6, including ERA5-ecPoint rainfall, ERA5 soil moisture, and static topographic variables. Each training configuration maintains the same temporal split, with 2016-2018 for training and 2019-2020 for testing.
The spatial divisions for Approaches 2 and 3 were selected to create regions with varying flash flood characteristics. The eastern region encompasses areas with frequent convective storms and urban flash flooding, the western region includes mountainous terrain with orographic effects, the northern region experiences both snowmelt-influenced and convective events, whilst the southern region is dominated by tropical moisture and intense convective systems.

\subsection{Model configuration}
The analysis employs the XGBoost model developed in Chapter \ref{data_driven_flash_floods_short_medium_range}, applying it to systematically modified training datasets to quantify the trade-offs between spatial coverage and observation density.
The XGBoost model parameters remain consistent with those optimised in Chapter \ref{data_driven_flash_floods_short_medium_range} to ensure that performance differences arise solely from training data modifications rather than model architecture changes. The hyper-parameter values for the XGBoost model considered in this chapter are presented in Table \ref{}. No additional regularisation or modification was applied to maintain comparability across approaches.

\subsection{Performance evaluation}
Model performance was evaluated using two primary metrics. The first metric analyses the \textit{distribution of predicted probabilities} obtained with the three different training approaches to assess how different training configurations affect the model's confidence in its predictions. For this scope, violin plots will be used as changes in distribution shape, spread, and the presence of intermediate probability values provide insights into model behaviour under different data availability scenarios. The second metric considers the \textit{flash flood detection capability}, which is quantified as the number of grid cells where the predicted flash flood probability exceeds the climatological average. 

\subsection{Global Application}
Following the sensitivity analysis, the best-performing approach was selected for global application. The regionally-trained CONUS model was applied to global ERA5 and ERA5-ecPoint fields to generate predictions of areas at risk of flash floods worldwide. Due to the absence of comprehensive global flash flood databases with sufficient density for quantitative verification, the global application was evaluated through case study analysis of major flash flood events reported in international media.


%%%%%%%%%%%%%%%%%
\section{Results}

\subsection{Extension of regional training to global application}

\begin{figure}[htbp]
\centering
\includegraphics[width=\textwidth]{sensitivity_analysis_global_extension.png}
\caption{\textbf{Global extension performance of regionally-trained XGBoost under different U.S. training data sampling strategies.} Violin plots show predicted probability distributions, whilst dots indicate percentage change in flash flood detection capability relative to climatological baseline. Three approaches tested: random sampling (Approach 1), domain-restricted training (Approach 2), and sub-regional training (Approach 3).}
\label{fig:sensitivity_analysis_global_extension}
\end{figure}

The global application of the regionally-trained XGBoost was evaluated through three distinct approaches, each employing different spatial sampling strategies for training data inclusion (Figure \ref{fig:sensitivity_analysis_global_extension}). The baseline configuration utilising all available flash flood reports in the Storm Event Database, within the verification period, produced a conservative probability distribution, with most predictions remaining below 10\% and 247,953 grid boxes exceeding the climatological average flash flood probability (0.08\%).

Approach 1, which implemented a random reduction of the flash flood reports to predetermined percentages (90\%, 50\%, and 10\%) uniformly, across the entire U.S. domain, demonstrated variable performance. The 90\% sampling configuration merginally increased the detection capability (i.e. number of flash flood reports exceeding the climatological average of flash flood events) by 1.42\%, while the 50\% and the 90\% reductions showed, respectively, a degradation of 9.56\% and 36.50\%, suggesting non-linear relationships between training data volumes and model performance.

Approach 2 explored a spatially-constrained training strategy where the model was trained only over a certain part of the domain where observations are available. The model did not see at all the other part of the domain, and it was applied to create predictions over the whole CONUS domain. This approach simulates a global training that considers only those areas of the world with good observations during training, but uses the model globally to create predictions over the continuous global domain. The reduction in the predicted probabilities was the smallest over the three approaches, with the biggest  reductions of -13.75\% and -19.96\% when the model was trained using the parts of the domain with the smallest number of flash flood reports i.e., the east and the north, respectively. When using the parts of the domain with the biggest number of reports, i.e., the west and the south, there was an increase of +1.42\% and a minimal reduction of -1.18\% of grid-boxes with a flash flood signal.

Approach 3 explores a similar spatially constrained training, where the model is trained over the whole CONUS but with observations only over a restricted part of the domain. This approach simulates a global training of the model over the whole global domain, and considering all available reports. This is the approach with the biggest reduction in predictive ability, with reductions ranging from -52.04\% when the east side was considered and -16.22\% when the south was considered, where training data is most geographically limited.

The varying performance across different training strategies highlights the critical importance of training data representativeness for successful global model deployment. These results show that the distribution of predicted probabilities and the predictive capabilities do change for different data reduction strategies. In all cases, the probability distributions remain highly skewed towards zero probabilities. Such high concentration near the zero value does not surprise as flash floods are rare events. However, the shape and the spread of the probability values varies greatly, with more compressed distributions (with probability values rarely exceeding 2\%) over those cases that train the model with very little training data (cases 4, 5, and 9). Such compression of the probability distribution suggests that the model becomes increasingly conservative and uncertain with less data. When training instead over regions with higher training data volumes (cases 1, 6, 8, 10, and 12), the model seems to be more willing to predict moderate-to-high probabilities. The shape of the distributions is also important. Overall, the shape does not change compared to the baseline (grey violin plot, case 0) with the training approach 1, while the shapes change for approach 2 and 3. The "bulges" in the middle of the distributions, primarily in approach 2, but also in approach 3 suggest the model learned to produce intermediate probabilities (2-6\%), indicating the model developed a more nuanced risk stratification, without losing the capability to predict larger probabilities unless the training dataset is very poor, as in case 5 and 9 represented by the west coast of the CONUS (the area with the least number of flash flood reports over the CONUS). 

The application of these three training strategies suggest that while global applications can maintain reasonable performance under certain conditions, data volumes play a crucial role in determining the transferability of regional flash flood forecasting models to global applications. Due to the presented results, approach 2 is selected for the global application of the CONUS-focused training, i.e., the regional training over the CONUS, using only the Storm Event Database, is applied to global fields to create predictions of areas at risk of flash floods over a continuous global domain. While it is not possible to run a robust verification analysis over the whole global domain due to the sparse density of global datasets like EM-DAT and DesInventar, section \ref{verif_case_study} will provide a selection of case studies over the CONUS and around the world to examine the performance of the regional and global training.


\subsection{Catalogue of flash flood events}
\label{verif_case_study}


%%%%%%%%%%%%%%%%%%%%%
\section{Discussions}


%%%%%%%%%%%%%%%%%%%%%
\section{Conclusions}