% Adding a coloured vertical edge to the pages in the chapter
\ClearShipoutPicture
\AddToShipoutPicture{%
  \AtPageLowerLeft{%
    \checkoddpage
    \ifoddpage
      \begin{tikzpicture}[remember picture,overlay] % Odd page → right edge
        \draw[line width=80pt, colour_chapter5] 
             (\paperwidth,0) -- (\paperwidth,\paperheight);
      \end{tikzpicture}%
    \else
      \begin{tikzpicture}[remember picture,overlay] % Even page → left edge
        \draw[line width=80pt, colour_chapter5] 
             (0,0) -- (0,\paperheight);
      \end{tikzpicture}%
    \fi
  }%
}

%%%%%%%%%%%%%%%%%%%%%%%%%%%%%%%%%%%%%%%%%%%%%%%%%%%%%%%%%%%
\chapter{Flash-flood-focused verification of rainfall-based 
predictions of areas at risk of flash floods}
\label{flash_flood_focused_verification_rainfall_based_ff}
\graphicspath{{chapter_05/figures}{chapter_05/tables}}
%%%%%%%%%%%%%%%%%%%%%%%%%%%%%%%%%%%%%%%%%%%%%%%%%%%%%%%%%%%

\underline{\textbf{Authors' contribution for this chapter:}} Fatima M. Pillosu designed the study, with advice from Hannah Cloke and Christel Prudhomme, obtained the datasets, carried out the analysis, and led the writing of the manuscript. All authors assisted with writing the manuscript. Overall, 90\% of the writing was undertaken by Fatima M. Pillosu.

\vspace{\baselineskip}

\section*{PREFACE}
\addcontentsline{toc}{section}{PREFACE}

The first main analysis chapter (Chapter \ref{flash_flood_focused_verification_rainfall_based_ff}) undertakes the flash-flood-focused evaluation of short- and medium-range post-processed probabilistic point-scale rainfall forecasts. Such forecasts have been shown to predict better than the raw ERA5 flash-flood-triggering localised extreme rainfall events \citep{Pillosu_2025a}. While such an improvement in the prediction of localised extreme rainfall should enable the improved detection of areas at risk of flash floods, a flash-flood-focused assessment is fundamental to determining the extent to which enhanced rainfall prediction accuracy translates into meaningful improvements in flash flood hazard identification. In this chapter, \textcolor{colour_chapter5}{research question 1 (RQ1) "Can post-processed global NWP rainfall forecasts successfully identify areas at risk of flash floods up to medium-range lead times?"} is answered. Moreover, the first research component of this thesis is addressed, i.e. \textcolor{colour_chapter5}{developing a flash-flood-focused verification framework for predictions of areas at risk of flash floods, designed to benchmark the capability of different predictive systems}. Hence, this evaluation establishes a performance benchmark against which more sophisticated modelling approaches - incorporating additional hydrological and topographical parameters - can be measured. Such comparative analysis is essential for determining whether the increased computational demands and data requirements of complex systems yield commensurate improvements in flash flood prediction accuracy, or whether simpler precipitation-based approaches provide sufficient utility for early warning applications.

\clearpage

\section*{ABSTRACT}
\addcontentsline{toc}{section}{ABSTRACT}

\clearpage



%%%%%%%%%%%%%%%%%%%%%%
\section{Introduction}


%%%%%%%%%%%%%%
\section{Data}


%%%%%%%%%%%%%%%%%
\section{Methods}


%%%%%%%%%%%%%%%%%
\section{Results}

%%%%%%%%%
\section{Assessment of rainfall-based predictions of areas at risk of flash floods}
\label{verif_rainfall_based_fc}

\subsection{Overall verification scores: frequency bias and area under the ROC curve}

\begin{figure}[htbp]
\centering
\includegraphics[width=\textwidth]{chapter_05/figures/rainfall_based_ff_verif_overall_scores.png}
\caption{\textbf{Overall verification scores for the rainfall-based forecasts of areas at risk of flash flood.} Panel (a) shows the frequency bias (solid lines) for 1-year (in red), 5-year (in purple), 10-year (in light green), 20-year (in cyan), 50-year (in blue), and 100-year return period (in green). The corresponding shaded areas represent the confidence intervals at 99\% confidence level. The inset box contains a zoomed-in version of the panel to show better the frequency bias values close to 1 (representing perfect bias). Panel (b) shows the area under the ROC curve.}
\label{fig:rainfall_based_ff_verif_overall_scores}
\end{figure}


\subsection{Discrimination ability}

All forecasts for rainfall events exceeding the 1-year return period threshold (Figure \ref{fig:rainfall_based_ff_verif_breakdown_scores_roc_1rp}) exhibit a discrimination ability superior to random chance, as the curves are above the diagonal reference line. A systematic degradation in discrimination ability is observed with increasing lead time, with the Area Under the ROC Curve (AROC) values ranging from 0.675 for the short-range forecasts (Figure \ref{fig:rainfall_based_ff_verif_breakdown_scores_roc_1rp}a) to 0.612 (\sim9\% reduction) for t+120 (day 5, Figure \ref{fig:rainfall_based_ff_verif_breakdown_scores_roc_1rp}f). Despite such a reduction, the forecasts show a good discrimination ability throughout the forecast horizon. Day 1 forecasts (t+24) show a higher discrimination ability than the short-range forecasts, and only from day 2 forecasts (t+48), the discrimination ability of the long-range forecasts goes below that of the reanalysis. The relatively narrow confidence intervals (at 99\% confidence level) suggest that the differences in skill between forecast configurations are statistically meaningful at the considered confidence level.

\begin{figure}[htbp]
\centering
\includegraphics[width=\textwidth]{chapter_05/figures/rainfall_based_ff_verif_breakdown_scores_roc_1rp.png}
\caption{\textbf{ROC curves for tp >= 1-year return period for the rainfall-based forecasts of areas at risk of flash floods built with ERA5-ecPoint.} Panel (a) shows the ROC curve (blue solid line) for the short-range predictions together with the confidence intervals (blue shaded area) at 99\% confidence level. Panels (b) to (f) refer to the long-range forecasts, for accumulation periods ending in t+24, t+48, t+72, t+96, and t+120, respectively. The pink dots refer to the probability threshold at which the frequency bias has the closest value to 1 (i.e., perfectly reliable forecast), while the orange dot shows the value of the frequency bias for the lowest probability threshold available in ERA5-ecPoint (i.e., the 99th percentile).}
\label{fig:rainfall_based_ff_verif_breakdown_scores_roc_1rp}
\end{figure}

The pink dot in Figure \ref{fig:rainfall_based_ff_verif_breakdown_scores_roc_1rp}a shows that perfect reliability (i.e. frequency bias equal to 1) is reach for probabilities <= 23\%. For the long-range forecasts, the probability thresholds at which perfect reliability is achieved is compatible to the short-range, being 27\% for all the lead times except t+120 which is 26\%. The frequency bias for the lowest probability threshold (i.e. 99th percentile or probability threshold equal to 1\%) in the short-range forecasts equals to 28. The frequency biases for the long-range forecasts are similar, falling between 31 and 33. 

Similar results are obtained for the 5-, 10-, 20-, 50-, and 100-year return periods.

\begin{figure}[htbp]
\centering
\includegraphics[width=\textwidth]{chapter_05/figures/rainfall_based_ff_verif_breakdown_scores_roc_5rp.png}
\caption{\textbf{ROC curves for tp >= 5-year return period for the rainfall-based forecasts of areas at risk of flash floods built with ERA5-ecPoint.} Similar to Figure \ref{fig:rainfall_based_ff_verif_breakdown_scores_roc_1rp}.}
\label{fig:rainfall_based_ff_verif_breakdown_scores_roc_5rp}
\end{figure}

\begin{figure}[htbp]
\centering
\includegraphics[width=\textwidth]{chapter_05/figures/rainfall_based_ff_verif_breakdown_scores_roc_10rp.png}
\caption{\textbf{ROC curves for tp >= 10-year return period for the rainfall-based forecasts of areas at risk of flash floods built with ERA5-ecPoint.} Similar to Figure \ref{fig:rainfall_based_ff_verif_breakdown_scores_roc_1rp}.}
\label{fig:rainfall_based_ff_verif_breakdown_scores_roc_10rp}
\end{figure}

\begin{figure}[htbp]
\centering
\includegraphics[width=\textwidth]{chapter_05/figures/rainfall_based_ff_verif_breakdown_scores_roc_20rp.png}
\caption{\textbf{ROC curves for tp >= 20-year return period for the rainfall-based forecasts of areas at risk of flash floods built with ERA5-ecPoint.} Similar to Figure \ref{fig:rainfall_based_ff_verif_breakdown_scores_roc_1rp}.}
\label{fig:rainfall_based_ff_verif_breakdown_scores_roc_20rp}
\end{figure}

\begin{figure}[htbp]
\centering
\includegraphics[width=\textwidth]{chapter_05/figures/rainfall_based_ff_verif_breakdown_scores_roc_50rp.png}
\caption{\textbf{ROC curves for tp >= 50-year return period for the rainfall-based forecasts of areas at risk of flash floods built with ERA5-ecPoint.} Similar to Figure \ref{fig:rainfall_based_ff_verif_breakdown_scores_roc_1rp}.}
\label{fig:rainfall_based_ff_verif_breakdown_scores_roc_50rp}
\end{figure}

\begin{figure}[htbp]
\centering
\includegraphics[width=\textwidth]{chapter_05/figures/rainfall_based_ff_verif_breakdown_scores_roc_100rp.png}
\caption{\textbf{ROC curves for tp >= 100-year return period for the rainfall-based forecasts of areas at risk of flash floods built with ERA5-ecPoint.} Similar to Figure \ref{fig:rainfall_based_ff_verif_breakdown_scores_roc_1rp}.}
\label{fig:rainfall_based_ff_verif_breakdown_scores_roc_100rp}
\end{figure}


\subsection{Reliability}

All forecasts for rainfall events exceeding the 1-year return period threshold (Figure \ref{fig:rainfall_based_ff_verif_breakdown_scores_rel_diag_1rp}) exhibit a systematic overprediction across all lead times, as shown by the reliability diagram being below the diagonal line. This indicates that when the model predicts a given probability, the observed frequency of flash flood events is consistently lower. For example, when the forecasts indicate a 50\% chance of having a flash flood event, the observed frequency ranges from \sim10\% in the short-range forecasts (Figure \ref{fig:rainfall_based_ff_verif_breakdown_scores_rel_diag_1rp}a), and between 10\% (for t+24, Figure \ref{fig:rainfall_based_ff_verif_breakdown_scores_rel_diag_1rp}b) and 2\% (t+120, Figure \ref{fig:rainfall_based_ff_verif_breakdown_scores_rel_diag_1rp}f) in the long-range forecasts. As seen in the ROC curves, the confidence intervals at 99\% are fairly narrow, suggesting that the differences between the reliability diagrams at different lead times are significant at the considered confidence level. However, the confidence levels increase with increasing forecast probabilities. As seen in the corresponding sharpness diagrams (inset boxes in all panels of Figure \ref{fig:rainfall_based_ff_verif_breakdown_scores_rel_diag_1rp}), such widening of the confidence intervals is likely due to the low number of forecasts issued with probabilities higher than ~25\% (when the total number of instances lies below 1000 samples). Such predominance of low probability forecasts suggests that the model (ERA5-ecPoint) rarely expresses high confidence in extreme event occurrence. A notable characteristic of all the reliability diagrams in the Figure \ref{fig:rainfall_based_ff_verif_breakdown_scores_rel_diag_1rp} is the sharp increase in observed frequency for the highest probability bins (i.e. 80\% to 100\%). This steep rise suggests that when the model does issue high probability forecasts, these correspond to genuinely extreme events, though such forecasts remain infrequent.

\begin{figure}[htbp]
\centering
\includegraphics[width=\textwidth]{rainfall_based_ff_verif_breakdown_scores_rel_diag_1rp.png}
\caption{\textbf{Reliability diagrams for tp >= 1-year return period for the rainfall-based forecasts of areas at risk of flash floods built with ERA5-ecPoint.} Panel (a) shows the reliability diagram (blue solid line) for the short-range predictions together with the confidence intervals (blue shaded area) at 99\% confidence level. Panels (b) to (f) refer to the long-range forecasts for accumulation periods ending in t+24, t+48, t+72, t+96, and t+120, respectively. The inset boxes show the corresponding sharpness diagrams.}
\label{fig:rainfall_based_ff_verif_breakdown_scores_rel_diag_1rp}
\end{figure}

The temporal evolution from Figure \ref{fig:rainfall_based_ff_verif_breakdown_scores_rel_diag_1rp}a to f reveals subtle changes in the forecasts' reliability characteristics with increasing lead time. Whilst the general pattern of overprediction persists, from day 2 forecasts (t+48) results more squashed into a flat line over very small observed frequencies, indicating that even though the model issues forecasts with high probabilities of exceeding the 1-year return period at longer lead times with a similar frequency of the short-range forecasts and the day 1 (t+24) long-range forecast, such forecasts do not necessarily correspond to an observed flash flood event.

The reliability diagrams get closer to the diagonal line (indicating perfect bias) as we increase the rainfall threshold to identify the flash flood events. Perfect reliability is observed for rainfall events exceeding the 10-, 20-year, and 50-year return period with probabilities below 5\% at day 1 forecast (t+24) for the 10-, 20-year return period, and 10\% for the 50-year return period.

\begin{figure}[htbp]
\centering
\includegraphics[width=\textwidth]{rainfall_based_ff_verif_breakdown_scores_rel_diag_5rp.png}
\caption{\textbf{Reliability diagrams for tp >= 5-year return period for the rainfall-based forecasts of areas at risk of flash floods built with ERA5-ecPoint.} Similar to Figure \ref{fig:rainfall_based_ff_verif_breakdown_scores_rel_diag_1rp}.}
\label{fig:rainfall_based_ff_verif_breakdown_scores_rel_diag_5rp}
\end{figure}

\begin{figure}[htbp]
\centering
\includegraphics[width=\textwidth]{rainfall_based_ff_verif_breakdown_scores_rel_diag_10rp.png}
\caption{\textbf{Reliability diagrams for tp >= 5-year return period for the rainfall-based forecasts of areas at risk of flash floods built with ERA5-ecPoint.} Similar to Figure \ref{fig:rainfall_based_ff_verif_breakdown_scores_rel_diag_1rp}.}
\label{fig:rainfall_based_ff_verif_breakdown_scores_rel_diag_10rp}
\end{figure}

\begin{figure}[htbp]
\centering
\includegraphics[width=\textwidth]{rainfall_based_ff_verif_breakdown_scores_rel_diag_20rp.png}
\caption{\textbf{Reliability diagrams for tp >= 5-year return period for the rainfall-based forecasts of areas at risk of flash floods built with ERA5-ecPoint.} Similar to Figure \ref{fig:rainfall_based_ff_verif_breakdown_scores_rel_diag_1rp}.}
\label{fig:rainfall_based_ff_verif_breakdown_scores_rel_diag_20rp}
\end{figure}

\begin{figure}[htbp]
\centering
\includegraphics[width=\textwidth]{rainfall_based_ff_verif_breakdown_scores_rel_diag_50rp.png}
\caption{\textbf{Reliability diagrams for tp >= 5-year return period for the rainfall-based forecasts of areas at risk of flash floods built with ERA5-ecPoint.} Similar to Figure \ref{fig:rainfall_based_ff_verif_breakdown_scores_rel_diag_1rp}.}
\label{fig:rainfall_based_ff_verif_breakdown_scores_rel_diag_50rp}
\end{figure}

\begin{figure}[htbp]
\centering
\includegraphics[width=\textwidth]{rainfall_based_ff_verif_breakdown_scores_rel_diag_100rp.png}
\caption{\textbf{Reliability diagrams for tp >= 5-year return period for the rainfall-based forecasts of areas at risk of flash floods built with ERA5-ecPoint.} Similar to Figure \ref{fig:rainfall_based_ff_verif_breakdown_scores_rel_diag_1rp}.}
\label{fig:rainfall_based_ff_verif_breakdown_scores_rel_diag_100rp}
\end{figure}



%%%%%%%%%%%%%%%%%%%%%
\section{Discussions}


%%%%%%%%%%%%%%%%%%%%%
\section{Conclusions}
